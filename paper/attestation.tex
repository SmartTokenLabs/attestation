\documentclass[11pt]{article} % \usepackage[utf8]{inputenc}
\usepackage[latin1]{inputenc}
\usepackage[american]{babel}
\usepackage{a4}
\usepackage{latexsym}
\usepackage{amssymb}
\usepackage{amsmath}
\usepackage{epsfig}
\usepackage[T1]{fontenc}
\usepackage{lmodern}
\usepackage{datetime}
\usepackage{wrapfig}

\usepackage{mathtools,amssymb}
\usepackage{amsthm}
\usepackage{float}
\usepackage{caption}
\usepackage{url}

\usepackage{verbatim}

\usepackage{graphicx}
\usepackage{wrapfig}
\usepackage{acronym}
\usepackage{makeidx}


\newtheorem{thm}{Theorem} 
\newtheorem{fact}{Fact} 
\newtheorem{example}{Example} 
\newtheorem{definition}{Definition} 
\newtheorem{corollary}{Corollary} 
\newtheorem{lemma}[thm]{Lemma}
\newtheorem{proposition}[thm]{Proposition}
%\theorembodyfont{\normalfont\sffamily}
\newtheorem{remark}{Remark} 
\newtheorem{conjecture}{Conjecture} 
\newtheorem{problem}{Problem} 
\newtheorem{assumption}{Assumption} 
\newtheorem{observation}{Observation} 
\usepackage{listings}
\usepackage{comment}
\usepackage{subfig}
\usepackage{graphicx} 
\usepackage{graphics}
\usepackage{txfonts}
\usepackage{algorithm,algorithmic}
\usepackage{todonotes}
\usepackage{epigraph}
\usepackage{afterpage}

\newcommand{\F}{\mathbb{F}}
\newcommand{\Hash}{\ensuremath{\mathrm{H}}}
\newcommand{\aut}{\ensuremath{\mathsf{Aut}_i}}
\newcommand{\att}{\ensuremath{\mathsf{att}_{i,j}}}
\newcommand{\val}{\ensuremath{\mathsf{val}_{i,j,l}}}
\newcommand{\usr}{\ensuremath{\mathsf{Usr}_l}}
\newcommand{\con}{\ensuremath{\mathsf{Con}_m}}

\newcommand{\lab}{\ensuremath{\mathsf{label}}}
\newcommand{\mpk}{\ensuremath{\mathsf{mpk}}}
\newcommand{\msk}{\ensuremath{\mathsf{msk}}}
\newcommand{\pk}{\ensuremath{\mathsf{pk}}}
\newcommand{\sk}{\ensuremath{\mathsf{sk}}}

\newcommand{\expireDate}{\ensuremath{\mathsf{expireDate}}}

\newcommand{\Gen}{\ensuremath{\mathrm{Gen}}}
\newcommand{\Der}{\ensuremath{\mathrm{Der}}}
\newcommand{\Sign}{\ensuremath{\mathrm{Sign}}}
\newcommand{\Ver}{\ensuremath{\mathrm{Ver}}}
\newcommand{\Addr}{\ensuremath{\mathrm{Addr}}}

\newcommand{\Sib}{\ensuremath{\mathrm{Sib}}}

\newcommand{\Bloom}{\ensuremath{\mathrm{Bloom}}}
\renewcommand{\insert}{\ensuremath{\mathrm{insert}}}
\newcommand{\query}{\ensuremath{\mathrm{query}}}

%\usepackage[cm]{fullpage}

\author{Weiwu Zhang and Tore K. Frederiksen} \title{Attestation on Ethereum}
\begin{document}
\maketitle
\begin{abstract}
We present an Ethereum-based solution to the problem of attestation in smart contracts without compromising a user's privacy. Specifically we present a solution to the problem where a user has certain values on different attributes (attested by an authority) and the user wishes to prove only \emph{one} of these values, to a smart contract. However, the user does not want an observer of the blockchain to be able to link these proofs together whether or not a user participates in the same of different smart contracts.
\end{abstract}


\section{Introduction}
Consider the setting where there exists an authority that wishes to attest to certain attributes of any number of users. For example, this authority could be a government and the users, its citizens. The government can then attest that a certain user was born before a certain year, or that the user was born in a certain city, or that it does not have a criminal record. We note that some of these attributes are immutable, but that they may also be mutable, e.g. the criminal status. A user then wishes to prove to a smart contract that he obeys a specific attribute, say that he was born before the year 2000 and thus is above 18 years of age. However, he might also want to prove to another smart contract that he does not have a criminal record. Later he might want to prove to another execution of the first smart contract that he was also born in Sydney. These proofs act as a prerequisite for the smart contract to execute.
% This means that such a proof cannot be carried out out-of-band as atomicity is required in order to fully leverage the power of reactivity of smart contracts. 
Thus we cannot simply have a user supply a one-time public address and a payment to the smart contract and then verify attestation out-of-band and wait for the authority to commit to the blockchain that it accepts the attestation. The problem is that this would imply that the user's assets are in limbo between its initial commit and the acceptance of the service. 

Instead, we might imagine that there exists a service verifying user's
attestation out-of-band. The user convinces the service to accept an
attestation on the user, then, the service sends a signed message
verifying this back to the user. The user could then input this, along
with its finances to the blockchain to start executing an atomic smart
contract transaction. The problem with this approach is that it
requires such a service to exist. Many smart contracts which may
require attestations does not require such a service.  Say if the
contract reflects a simple gambling game where the finances are shared
amongst the user executing the contract, like ``King of the
Ether''. Furthermore it breaks the "Code is law" paradigm since the
execution of the smart contract would depend not only on the user
being able to attest the attributes, but also the verifying service is
willing to and able to sign an approval.

This problem is in itself not too hard to handle by a user simply publishing all its attested attributes. However, we require that the different attributes cannot be linked to a particular user. We require this \emph{even} if the user is communicating with the same smart contract at different points in time, not just that the user is unlinkable between different attributes and different contracts.

A general prerequisite for this is, with the lack of built-in sensorship resistance, that for every transaction the user wishes to keep unlinkable it uses a different public address for financial transfer. 

Finally, we note that there can be distinct authorities with the same users but attesting to different attributes.

\subsection{Unsuitable solutions}
The cryptographic theory community has a theoretical construct, called an \emph{attribute-based signature scheme}~\cite{MPR11} which functions as follows: An authority holds a public key and can issue different private keys for users (associated with the authority's public key). Furthermore, the authority can associate a set of attributes with the private key of a key pair. It is then possible for the holder of the private key to sign a message and include a predicate of its choosing, based on the attributes associated with the key. The public key can then be used to verify the signature and the validity of the predicate. In particular, only the validity of the predicate is given away, and not the specific values of the attributes. Furthermore, seeing several signatures it is not possible to determine if they have been constructed using the same private key. 

With such a scheme we can have the authority to issue private keys to a given user based on the attributes which it holds. This functions as the user's attestation. Now, when a user wishes to prove some predicate on its attributes it simply signs the public address it wishes to use and inputs this to a smart contract, which is designed to verify this signature using the authority's public key.

Unfortunately, such attribute-based signature schemes are still relatively new and not implemented in a product-ready way. Furthermore, they rely on rather heavy mathematical computations which are currently not natively supported in an Ethereum smart contract.

\section{Preliminaries}
We here define the different primitives required for our construction. First, assume we have a hash function hashing an arbitrary length input into a $\kappa$-bit digest. That is, $\Hash:\{0,1\}^*\to \kappa$. We use $\|$ to denote concatenation of two strings. We use a comma separated list within parentheses to denote an unspecified binary encoding of elements, i.e. $(a, b, c)$ is actually a bitstring encoding the values $a$, $b$, and $c$ in some deterministic manner. 

\subsection{Merkle tree}
Given a list of elements, $l_1, l_2, \dots, l_n$, a \emph{Merkle tree} is a balanced binary tree built on top of the list of elements, s.t. the $i$'th element of the list gets associated with the $i$'th leaf on the highest level; enumerating from left to right. Thus the $i$'th leaf will have a label equal to the content of $l_i$. Thus the Merkle tree will have height $h=\lceil \log_2(n) \rceil$ and all leafs (values $l_1, \dots, l_n$) are at level $h$. The label of each internal node is defined to be the hash digest of the concatenation of the labels of its children. That is, when enumerating nodes in level-order, letting the root have number 1, then the label of the $i$'th node is $\Hash((n_{2i}, n_{2i+1}))$, where $n_{i}$ denotes the label of the $i$'th node. 

% For convenience we will denote the sibling of a node $n$ by $\Sib(n)$. That is if $n$ is node number $i$  and a left child then $\Sib(n)$ will be be node number $i+1$ and thus a right child. On the other hand if $\Sib(n)$ is node $i$ and a right child, then $\Sib(n)$ is the left child and thus node $i-1$.

\subsection{Bloom filter}
A Bloom filter is a data structure used for efficient membership testing. The data structure is efficient in both testing and insertions. However, it does not support deletions. Furthermore, the filter has an upper bound of elements it can contain while keeping the probability of a false positive below a certain threshold.

We define a Bloom filter to consist of a $\lambda$ bit array and $\mu$ hash functions; each mapping an arbitrary length input to an integer between $1$ and $\lambda$. We generate such a filter with the method $\Bloom(\lambda, \mu)\to b$, where $b$ is a $\lambda$ bit array, initially 0-initialized. To insert an element $x$ into the filter, it is hashed by each of the $\mu$ hash functions and viewing each integer returned by these hash functions as an index into $b$, the bit with this index is set to 1. We denote insertion into $b$ by $b.\insert(x)$. 

It is now possible to check membership in a similar manner to insertion; basically to check if $x$ is in the filter, hash it using each of the hash functions and using the integer returned as an index into $b$, check that each bit of these indexes is set to 1. If it is, then it is probably in the filter. We denote the query method by $b.\query(x)$, and say it returns $\top$ if $x$ is in the filter with large probability and $\bot$ if $x$ is definitely not in the filter. 


\subsection{Derived signature scheme}
%BIP32
We require a derivable signature scheme. Formally this is defined as a tuple of algorithms $(\mathrm{Gen, Der, Sign, Ver})$. Here $\mathrm{Gen}$ is a parameter generation algorithm, generating a master public and private key, denoted by $\mpk$, respective $\msk$. $\mathrm{Der}$ is an algorithm for deriving a new key, taking either the master public key or the master private key as input, along with an offset value, $\delta$. It also takes an enum value as input, describing whether it should derive a new public key, or private key, and whether the master key given as input is public or private. The algorithm then derives a new key deterministically based on $\delta$.  We describe these algorithms in more detail below:
\begin{description}
	\item[Generation] ~ An algorithm $\mathrm{Gen}(1^k) \to (\mathsf{mpk}, \mathsf{msk})$ taking as input a unary representation of a security parameter returns a pair of values, where $\mpk$ represents a \emph{master public key} and $\msk$ represents a \emph{master private key}. 
	\item[Derive] ~ An algorithm $\mathrm{Der}(\mathsf{direction}, \delta, \mathsf{key})\to \delta\mbox{-}\mathsf{key}$ where $\mathsf{direction}$ is either $\mathsf{public}\mbox{-}\allowbreak\mathsf{to}\mbox{-}\mathsf{public}$ or $\mathsf{private}\mbox{-}\mathsf{to}\mbox{-}\mathsf{private}$. $\delta\in \{0,1\}^*$ is an offset which uniquely defines the derived key based on the master key.  $\mathsf{key}$ is the master key used to produce the derived key $\delta\mbox{-}\mathsf{key}$.
	\item[Sign] ~ An algorithm $\mathrm{Sign}(x, \mathsf{sk})\to t$ taking as input a message $x\in\{0,1\}^*$ along with a secret key $\mathsf{sk}$ and derives a tag $t\in\{0,1\}^*$.
	\item[Verify] ~ An algorithm $\mathrm{Ver}(x, t, \mathsf{pk}) \to b$ takes as input a message $x\in \{0,1\}^*$, a tag $t\in \{0,1\}^*$, along with a public key $\mathsf{pk}$ and return a bit $b\in\{\top, \bot\}$ indicating whether the tag $t$ was constructed on the message $x$ using the secret key associated with the public key $pk$. In particular $\mathrm{Ver}(x, \mathrm{Sign}(x', \mathsf{sk}), \mathsf{pk})\to \top$ if and only if $x=x'$ and $\mathrm{Der}(\cdot, \delta, \mathsf{mpk})\to \mathrm{pk}$ and $\mathrm{Der}(\cdot, \delta, \mathsf{\msk})\to \mathrm{sk}$ where $\mathrm{Gen}(1^k)\to (\mathsf{mpk}, \mathsf{msk})$ except with negligible probability in $k$ for any $\delta\in\mathbb{N}$.
\end{description}
In relation to this we associate an algorithm $\Addr (\delta\mbox{-}\pk)\to a$ which takes a derived public key as input and derives a valid Ethereum public address, $a$.

\section{Protocol Design}
We consider the setting where there is an ever expanding set of authorities, where the $i$'th authority is denoted by $\mathsf{Aut}_i$ which has a public key pair $\pk_{\aut}, \sk_{\aut}$. Each authority is able to attest certain attributes. We let the attributes associated with $\mathsf{Aut}_i$ be denoted by the set $\{\mathsf{att}_{i,j}\}_{j=1, \dots, \beta}$. We also assume there is an independent, ever-expanding set of users, where the $l$'th user is denoted by $\mathsf{Usr}_l$. Finally, we consider a third, independent and ever expanding set of smart contracts where the $m$'th smart contract is denoted by $\mathsf{Con}_m$.

The overall idea of the protocol is as follows:
\paragraph{Setup.} The user $\usr$ proves values of the attributes to an authority $\aut$ out-of-band. $\aut$ then constructs $\alpha$ one-time attestations of the user's values. Each of these attestations consists of an Ethereum address constructed from a derived public key, based on the user's master public key. An attestation consists of a Merkle tree where each leaf is associated to one of the attributes the authority can attest. The value attested to, along with expiration date and any other meta data, is then hashed with a random seed (for each one-time attestation, but derived from a single ``master seed'') and this will constitute a leaf in a Merkle tree. Finally the root of the Merkle tree is concatenated with a fresh Ethereum address and the digest is then signed by the authority. The authority then sends only the signatures and the master seed to the user. Furthermore, it internally stores the user's master public key, the master seed, and attribute values, in case it needs to do revocation later. The user stores \emph{all} the signatures, but not the Merkle trees.

\paragraph{Revoke}
Each smart contract $\con$ working on attestations from $\aut$ stores internally, in an ``updateable'' region, a Bloom filter. This Bloom filter will keep track of all the attestations which have not expired yet, but have been revoked. This filter will initially be empty, but once it becomes necessary for $\aut$ to revoke it, it will put the message it signed (the root from the Merkle tree) into the Bloom filter. $\aut$ will then update the smart contract so it contains the updated Bloom filter. The authority will internally keep track of all the attestations in the \emph{current} Bloom filter so that whenever it is updated, it doesn't reinsert expired attestations.

\paragraph{Proof}
The user $\usr$ keeps an index of how many one-time attestations it has used. When it wishes to prove that it has an attestation from $\aut$ attesting to it having value $\val$ on attribute $\att$, it picks the next unused index in the list of one-time attestations. Then based on this index, the master salt, and its attested values, $\usr$ recomputed the Merkle tree for this index. It then constructs a list containing the content of the nodes from the leaf based on $\att$, up to the root. However, it also includes the siblings to each of the nodes in the list. It then inputs this list, along with the derived public key for this one-time attestation, to the smart contract. The smart contract can then verify the value attested to it as expected and then check the signature by recomputing the values of the path from the leaf to the root. Then, based on the root node and the current derived public key, the contract can derive the attestation which was actually signed by the $\aut$ and thus verify that the signature supplied by $\usr$ is valid.

In the sequel, we define these steps more formally.

\subsection{Setup}
A user $\usr$ wishes to receive an attestation from authority $\aut$, attesting certain values on $\beta$ attributes, $\{\att\}_{j=1, \dots, \beta}$. Denote the value of attribute $\att$ associated with user $\usr$ by $\val$. Assume the user holds a master public key and a master secret key, $\mpk_l$, respectively $\msk_l$, generated using $\mathrm{Gen}$. Finally assume that the user and the authority agree on an upper bound, $\alpha$, of the amount of proofs the user can do. 

To do this the user executes a necessary proof out-of-band to convince the authority which values it should have on the attributes. The parties then proceed as follows, assuming that $\usr$ and $\aut$ have an encrypted and authenticated channel of communication:
\begin{enumerate}
	\item $\usr$ sends $\mpk_l$ to $\aut$.
	\item $\aut$ picks a uniformly random salt $s_{i,l}\in\{0,1\}^\kappa$ and for $\iota=1, 2, \dots, \alpha$ computes the label $\lab_{i,j,l,\iota}=\left(\val, \expireDate, \Hash((s_{i,l}, i, j, l, \iota)\right)$ associated with $\att$ for each $\iota$.
	\item For each $\iota$ enumerate the labels on $j$. I.e. as the list $\lab_{i,1,l,\iota}, \lab_{i,2,l,\iota}, \dots$. The authority then builds a Merkle tree on top of these labels, i.e. having the labels be the leafs of the tree. Denote the nodes (i.e. the digest computed) as $n_{i,l,\iota, \eta}$, where $\eta$ is the node index. Thus the leaves are indexed by $2^{\lceil\log_2(\beta)\rceil}, \dots, 2^{\lceil\log_2(\beta)\rceil}+\beta-1$ and the root by 1.
	% root node of the Merkle tree by $r_{i,l,\iota}$.
	\item The authority computes $\Der(\mathsf{public}\mbox{-} \mathsf{to}\mbox{-} \mathsf{public}, n_{i,l,\iota, 1}, \mpk_l)\to \pk_{i,l,\iota}$.
	\item The authority computes the signature $\Sign\left( \Hash\left(\Addr(\pk_{i,l,\iota}), n_{i,l,\iota, 1}\right), \sk_{\aut}\right)\to t_{i,l,\iota}$.
	\item The authority sends $(\{t_{i,l,\iota}\}_{\iota = 1, \dots, \alpha}, s_{i,l})$ to $\usr$.
	\item \usr stores the values $(\{t_{i,l,\iota}\}_{\iota = 1, \dots, \alpha}, s_{i,l})$ locally.
\end{enumerate}
We note that the server stores $\{\val\}_{j=1, \dots, \beta}$, $s_{i,l}$, $\mpk_l$ and the expiration date associated with the attestation given to user $\usr$ for use in case of revocation.

\subsection{Revocation}
An authority, $\aut$, wishes to revoke the attestation for a user, $\usr$, since attributes have changed or and been falsely given. For this purpose, we make sure that there is an ``updateable'' area of the $\con$ to store an initially empty Bloom filter $b_m$. The parameters are set on this filter s.t. the chance of a false positive is sufficiently small. With this filter $\aut$ associates a list of attestations that have been revoked. To revoke the attestations for user $\usr$ with master public key $\mpk_l$ then $\aut$ proceeds as follows:
\begin{enumerate}
	\item Create a new Bloom filter, denote it by $b_m'$.
	\item Take the current Bloom filter on $\con$, denoted by $b_m$. $\aut$ then look up internally the list of values associated with $\con$ and $b_m$. Denote these by $a_{m, 1}, \dots, a_{m, \rho}$.
	\item For each $q=1, \dots, \rho$ where the expiration date associated to $a_{m,1}$ has not expired. $\aut$ computes $b_m'.\insert(a_{m, q})$ and internally associates $a_{m,q}$ with $b_m'$.
	\item \sloppy For $\iota=1, 2, \dots, \alpha$ $\aut$ recomputes all the Merkle trees it signed using $\{\val\}_{j=1, \dots, \beta}$, $s_{i, l}$, and $\mpk_l$ (like it did during \textbf{Setup}). It then computes $a_{m, \iota}' = \Addr\left(\Der\left(\mathsf{public}\mbox{-}\mathsf{to}\mbox{-}\mathsf{public}, n_{i,l,\iota, 1}, \mpk_l\right)\right)$, $b_m'.\insert(a_{m, \iota}')$ and internally associates $a_{m,\iota}'$ with $b_m'$.
	\item $\aut$ updates $\con$ to use $b_m:=b_m'$.
\end{enumerate}
Note that unexecuted contracts in the pipeline before the revocation happens will not be executed. This can be seen as a feature. Also notice that privacy is not promised to a user who gets its attestation revoked. This is so since it is probably possible to do statistical analysis on the new Bloom filter as \emph{all} the user's public addresses get revoked at once. This is not a problem as a user who gets its attestation revoked has not behaved nicely and thus is not entitled to privacy.

We note that it is not necessary to construct a new Bloom filter every time a revocation happens, but it is sufficient to simply update the existing one to contain the revoked attestations. However, it must occasionally be renewed as it will otherwise get oversaturated and thus yield too many false positives. We keep track of the expiration dates of the attestations so that we can remove obsolete entries whenever we update the filter, thus preventing it from growing indefinitely.

\subsection{Proof}
A user $\usr$ wishes to prove to a smart contract $\con$ that the authority $\aut$ has attested that it has a certain value $\val$ on an attribute $\att$. To do so it proceeds as follows:
\begin{enumerate}
	\item $\usr$ picks the next unused $\iota$ and checks if $n_{i,l, \iota, 1}$ is in the Bloom filter. If it discards this $\iota$ and moves to the next unused $\iota$. Repeat until an unused $\iota$ is found where $n_{i,l, \iota, 1}$ is not in the Bloom filter.
	\item $\usr$ picks the next unused $\iota$ and for each attribute, $\att$, with $\val$ associated, computes the labels $\lab_{i,j,l,\iota}=(\val, \Hash((s_{i,l}, i, j, l, \iota))$.
	\item $\usr$ then enumerate these labels on $j$. I.e. as the list $\lab_{i,1,l,\iota}, \lab_{i,2,l,\iota}, \dots$. It then builds a Merkle tree on top of these labels. Denote the root node of the Merkle tree by $n_{i,l,\iota}$.
	\item The user then recovers the path from the leaf to the root of the Merkle tree, of the attribute which it wants to show an attestation on. That is, say it wishes to show the value on attribute $\att$ is $\val$; letting $\tau=2^{\lceil\log_2(\beta)\rceil}+j-1$, it then stores the list $\gamma=(\lab_{i,j,l,\iota}, \{n_{i,l, \iota, \eta}\}_{\eta =\lfloor\tau/2^0\rfloor, \Sib(\lfloor\tau/2^0\rfloor), \lfloor\tau/2^1\rfloor, \Sib(\lfloor\tau/2^1\rfloor), \lfloor\tau/2^2\rfloor, \Sib(\lfloor\tau/2^2\rfloor),\dots, 1)}$.
	\item $\usr$ then computes $\Der(\mathsf{public}\mbox{-}\mathsf{to}\mbox{-}\mathsf{public}, n_{i,l,\iota, 1}, \mpk_l)\to \pk_{i,l,\iota}$.
	\item $\usr$ recovers the signature $t_{i,l,\iota}$ it got from $\aut$ and inputs $(\gamma, \Addr(\pk_{i,l,\iota}), t_{i,l,\iota})$ to the smart contract $\con$.
	\item Mark the $\iota$ as used.
\end{enumerate}

We note that the smart contract $\con$ must be written s.t. it verifies that the attribute value of the user is as expected and that the signature of the authority is valid before executing. I.e. letting $\tau=2^{\lceil\log_2(\beta)\rceil}+j-1$, that 
\begin{align*}
n_{i, l, \iota, \tau } &= \Hash( \lab_{i,j,l,\iota} ) &&\wedge\\
n_{i, l, \iota, \eta} &= \Hash( (n_{i,l,\iota,  2 \eta}, n_{i,l,\iota, 2 \eta+1}) ) \quad \text{for } \eta= \lfloor\tau/2^1\rfloor, \lfloor\tau/2^2\rfloor, \dots, 1) &&\wedge\\
\top&=\Ver(\Hash(\Addr(\pk_{i,l,\iota}), n_{i,l,\iota, 1}), t_{i,l, \iota}, \pk_{\aut}) &&\wedge\\
\expireDate &\geq \text{Current function call time} &&\wedge\\
\bot &= b_m.\query(n_{i, l, \iota, 1})
\end{align*}

\section{Security}
The scheme relies on the security of the underlying signature scheme, in particular when repeatedly deriving new keys based on a master key pair. However, this is no different than what is generally required by Ethereum. The only other cryptographic construction that is added on top is the hash function. We require this to be second-preimage resistant to protect against a corrupt user trying to fake an unattested value. We also require it to be hard to find a preimage to protect against anyone viewing the blockchain trying to learn some auxiliary info on users attribute values, i.e. being able to inverse the digest of one of the leaf labels.

The scheme assumes that the authority is not corrupt, but prevents a malicious user from faking attestations or changing attestations received by the authority. The scheme furthermore protects an honest user's privacy. In particular, any usage of an attestation is unlinkable to any other usage of the attestation. The exception occurs if the attestation needs to be revoked. In this case, privacy \emph{may} leak to someone analyzing the difference between the Bloom filter before revocation and after revocation. We finally note that the user's private key will never leave the user and thus even if the authority is corrupt, or if the user gets its attestation revoked, its private key will still be protected.

\subsection{Attribute values}
Considerations must be made when selecting which type of values to store in the attributes. In particular person-identifying information such as names and birthdays might leak more than desirable. When possible it is better to have attribute values asserting membership of a large group. I.e. instead of birthday, an age interval might be more appropriate. That is, the attribute contains a boolean value stating whether the person is born more or less than 18 years ago. 


\bibliographystyle{alpha}
\bibliography{bibliography}
\end{document}

attributes are associated with server and not contract
outline of variables
